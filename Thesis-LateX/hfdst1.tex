%Het eerste hoofdstuk van je thesis.
\chapter{Inleiding}
De computerwereld maakt de laatste jaren grote stappen op vlak van Machine Learning\cite{Minar18}. Deze vorderingen werden gedreven door onder meer de nieuwste ontwikkelingen op vlak van computerrekenkracht en de vierde industri\"ele revolutie\cite{bloem2014fourth}. Door de veelvuldige toepassingsmogelijkheden werd \gls{ml} een populair en veelbesproken onderwerp. Tegenwoordig kan een bepaalde vorm van \gls{ai} in elke sector teruggevonden worden\cite{russell2016artificial}. Van hartritmestoornissen in de medische sector tot commentaarherkenning in de toeristische stiel. 

Een branche in de \gls{ml} die steeds meer in de schijnwerpers staat, is in de logistieke sector\cite{barreto2017industry}. Door de steeds verder doorgedreven automatisatie van bedrijven, wordt er ook in bijvoorbeeld magazijnen geopteerd voor het optimaliseren van onder meer het leveren van de verschillende onderdelen en de veiligheid in het magazijn.. Gebruik maken van zelfrijdende vorkheftrucks is een mogelijke optie in het verbeteren van de effici\"entie. Niet alleen in het magazijn maar ook op de weg is er een groeiende belangstelling naar zelfrijdende auto's, die ontwikkeld worden door grote bedrijven zoals Tesla en Uber. In beide cases zal er een zekere vorm van positiebepaling nodig zijn. Het is van groot belang dat deze bepaling zo accuraat mogelijk plaats vindt, met niet alleen een juiste locatie, maar ook op zo kort mogelijke tijdsperiode. Deze nood aan \textit{low latency} kan het verschil betekenen tussen een voertuig die beslist dat hij moet vertragen of beslist dat hij veilig kan doorrijden maar toch een botsing veroorzaakt. De berekening van die cruciale locatiebepaling kan zowel in de \textit{cloud}, als in de \textit{edge}\cite{edgecomputingLi} gebeuren en gebeurt vooral met behulp van \gls{ml}. Hierbij wordt met cloud verwezen naar het verwerken van data op een locatie ver weg van het voertuig zoals serverzalen. Met edge wordt dan weer een locatie dichtbij het voertuig bedoeld. Dit kan zowel op als vlakbij het voertuig zijn. Doordat \textit{cloud computing} een toegevoegde latency teweeg brengt van meerdere tientallen milliseconden, zal de keuze voor \textit{edge computing} vallen. Indien de berekeningen in de edge plaats vinden zal de afstand tussen re\"ele en virtuele locatie kleiner zijn, wat leidt tot een betere positiebepaling. Deze heeft dan weer tot gevolg dat inschattingen accurater plaats vinden en ongevallen vermeden kunnen worden. Welke hardware men gebruikt kan vari\"eren van applicatie tot applicatie. De berekening zelf wordt uitgevoerd met behulp van \gls{ai}. We maken hier gebruik van doordat \gls{ai} verschillende baten heeft. Deze voordelen worden besproken in de volgende paragraaf.

In deze thesis zal men trachten om in de edge-kant een benchmark op te stellen waarmee we verschillende \gls{sbc}s met elkaar vergelijken. Dit instrument moet meer inzicht verschaffen in welke mate machineleertechnieken toepasbaar zijn. Hoe groot is de latency die optreedt? Welk verbruik en complexiteit van het netwerk gaat er hier mee gepaard? De benchmark zal toegepast worden op verschillende Single Board Computers. Ook tussen deze hardware-opties wordt er een afweging gemaakt. Welk toestel is voordeliger in welke situatie? Is een goedkoper toestel tot evenwaardige resultaten in staat?

