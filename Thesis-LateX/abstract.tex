\textbf{Benchmark voor latency bij edge-devices.}

Er is een hele grote verscheidenheid aan onderwerpen waarbij de duur van verkrijgen van resultaten binnen zekere tijdslimieten moeten vallen. Hiervoor wordt vaak een afweging gemaakt tussen cloud en edge computing. In de cloud kan met krachtige en snelle hardware worden gewerkt. Echter treedt hierbij een significante vertraging op, veroorzaakt door het verzenden van data over het internet. Uitvoeren van programma's in de edge hebben geen last van deze vertraging. Dit kan voordelig zijn bij latency-gevoelige applicaties zoals het zelfstandig rijden van voertuigen. 

% Eerste paragraaf: waarover gaat de scriptie, geef een eenvoudige beschrijving van de doelen en methodes
Deze scriptie bespreekt een benchmark van drie verschillende edge-toestellen en een reguliere computer. Deze benchmark biedt informatie over de latency en processor-performantie van elk device. Deze zal meer inzicht bieden wanneer men een kosten-baten analyse maakt. De onderzochte devices binnen deze thesis zijn de toestellen zoals de Google Coral Dev die over zo wel een CPU als TPU bezit en de Raspberry Pi die enkel over een CPU beschikt. Ook wordt een Nvidia Jetson Nano en een Personal Computer gebruikt die zowel een CPU als een GPU bezitten.

% Tweede paragraaf: geef uitleg over de scriptiestructuur en vertel iets over de inhoud
Om een representatieve benchmark te bekomen is het nodig om verschillende programma's op een identieke wijze te testen. De te beproeven programma's zullen verdeeld worden over de categorie\"en regressie en classificatie. De data die uit de benchmark afgeleid wordt bestaat uit de latency en het processor-gebruik van het runnen van elk programma. Deze worden dan herwerkt en genormaliseerd om de resultaten eenvoudiger te analyseren. Hierbij worden ook factoren zoals prijs, kloksnelheid en energieverbruik in rekening gebracht.

% Derde paragraaf: het besluit, met een kort overzicht van de resultaten
Uit de resultaten kan er een duidelijk verschil afgeleid worden tussen de verschillende edge-toestellen. Hierbij komt de Nvidia Jetson Nano het best naar voor. Het toestel behaalt de laagste latency voor alle toegepaste programma's. De resultaten geven ook aan dat de Coral Dev effici\"enter met energie omspringt. Voor het uitvoeren van hetzelfde programma zal de Coral Dev minder energie verbruiken.


\underline{Trefwoorden:} benchmark, edge-devices, latency, Jetson Nano, Coral Dev, Raspberry Pi