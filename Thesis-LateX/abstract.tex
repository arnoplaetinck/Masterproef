\textbf{Benchmark voor latency bij edge-devices.}

Een van de meer spraakmakende nieuwe technologie\"en van vandaag is het zelfstandig rijden van voertuigen zonder hulp van een extern persoon. Deze maakt gebruik van Artifici\"ele Intelligentie (AI) om autonome beslissingen te nemen over het versnellen en vertragen van het rijtuig. De berekeningen die aan de basis van deze beslissing liggen, vereisen een uiterst accurate locatiebepaling. Hiervoor is de tijdsduur van het berekenen van de beslissing van groot belang.

% Eerste paragraaf: waarover gaat de scriptie, geef een eenvoudige beschrijving van de doelen en methodes
Deze scriptie bespreekt een benchmark van drie verschillende edge-toestellen en \'e\'en non-edge-toestel. Deze benchmark biedt informatie over de latency en performantie van elk device. Bovendien kan er een vergelijking tussen GPU- en CPU-toestellen gemaakt worden. Door de verkregen informatie heeft men meer inzicht bij het kiezen voor een bepaald toestel.

% Tweede paragraaf: geef uitleg over de scriptiestructuur en vertel iets over de inhoud
Om een representatieve benchmark te bekomen is het nodig om verschillende programma's op een identieke manier te testen. De te beproeven programma's zullen verdeeld worden over de categorie\"en regressie en classificatie. De data die uit de benchmark afgeleid wordt bestaat uit de latency en het processor-gebruik van het runnen van elk programma. Deze worden dan herwerkt en genormaliseerd om de resultaten duidelijker te analyseren.

% Derde paragraaf: het besluit, met een kort overzicht van de resultaten
Uit de resultaten kan er een duidelijk verschil afgeleid worden tussen de verschillende edges-toestellen. Hierbij komt de Nvidia Jetson Nano het best naar voor ondanks de hogere kostprijs en een hoger energieverbruik. Wel kan er nog een argument gegeven worden voor de Google Coral Dev die in specifieke programma's met geoptimaliseerde Tensorflow Lite modellen een voordeel kan halen.











\underline{Trefwoorden:} benchmark, edge-devices, latency, Jetson Nano, Coral Dev