\chapter{literatuustudie VLP}
%intro
\section{\acrfull{vlc}}
%RF bandwidths are failing to meet requirements, VLC is promising alternatives
%korte intro
De \acrfull{led} is een technologie dat gebruikt kan worden voor verlichting als communicatie. Hierdoor is het geschikt voor optische draadloze communicatie in de vrije ruimte. \gls{vlc} is een technologie dat hier op bouwt en veel verwachting heeft naar toekomstige applicaties toe. 
\section{\acrfull{vlp}}
Naast communicatie is lokalisatie nog een belangrijke toepassing met een groot potentieel. Op plekken waar traditionele lokalisatiemethoden zoals GPS falen, zoals indoor scenario's, is men op zoek naar een technologie die wel kan voldoen aan de voorschriften. \acrfull{vlp} is naast o.a. \acrfull{uwb}, \acrfull{rfid}, Wi-Fi en fingerprint een uitdager om deze rol in te vullen. Het nadeel van deze laatste technologie\"en is dat er nood is aan het installeren van een extra \acrfull{ap} en apparatuur. Dit leidt tot stijgende kosten voor het installeren, gebruik en onderhoud.
\gls{vlp} is veel belovend juist omdat het deze extra kosten kan vermijden door de duale rol van de apparatuur. De \glspl{led} zorgen voor zowel verlichting als lokalisatie in dezelfde behuizing, door slechts één installatie en heeft dezelfde noden als gewone verlichting. In komende paragrafen leggen we verschillende onderdelen en technieken van \gls{vlp} uit.
\subsection{Indoor Positioning Techniques}
\gls{vlc}-based-\gls{ips} kan op verscheidene manieren uitgevoerd worden. In de paragrafen hieronder bespreken we hoe de verschillende technieken ingevuld kunnen worden. We bespreken onder andere \gls{led} technologie, modulatiemethodes, type ontvangers en classificatie van \gls{vlc}-based-\gls{ips} op basis van 3 sleuteleigenschappen.

\subsubsection{\gls{led} Technologie}
Door het lage vermogensverbruik en de lange levensduur is \gls{led} een veel gebruikt middel om verlichting van ruimtes te bereiken. Er zijn twee veelgebruikte \gls{led}-technologie\"en om wit licht te produceren: 

\begin{itemize}
	\item \textbf{White \glspl{led}:} Bij dit ontwerp cre\"ert men wit licht door een gele fosfor over een blauwe \gls{led} te plaatsen. De combinatie van zowel de gele fosfor als het blauwe uitgestraalde licht zorgt dat ons oog de stralen als wit licht interpreteert. De eenvoud laat toe om de prijs en de complexiteit laag te houden, waardoor dit de meest gekozen techniek is.
	
	\item \textbf{RGB-\glspl{led}:} Door de combinatie van de kleuren rood, groen en blauw is het ook mogelijk om een wit licht te cre\"eren. Deze optie vraagt wel om twee extra \glspl{led} waardoor de kostprijs stijgt. Het gebruik van de RGB-\glspl{led} heeft wel het voordeel dat er aan \gls{csk} kan gedaan worden. Deze methode heeft dus een bredere functionaliteit.
\end{itemize}

\subsubsection{Modulatiemethode}
Doordat de \glspl{led} voor zowel verlichting als communicatie dienen is van uiterst belang dat voorzieningen voor het ene, het andere niet in gedrang brengen. De gekozen modulatiemethode voor communicatie moet dus ondersteuning bieden zodat de werking van de verlichting normaal kan blijven verlopen. Om hier aan te voldoen moet er dus ondersteuning zijn voor dimming en flicker controle.
\begin{itemize}
	\item \textbf{\acrfull{ook}:} \gls{ook} is de simpelste vorm van \acrfull{ask} modulatie waarbij de \gls{led} aan en uit wordt gezet. De eenvoudige implementatie is de reden waarom de techniek wijd gebruikt wordt in de draadloze communicatie. Er kan een onderscheid gemaakt worden tussen \acrfull{nrz} \gls{ook} en \acrfull{rz}\gls{ook}. Bij \gls{rz}\gls{ook} zal het signaal terug naar 0 gaan achter elk uitgezonden symbool. Dit zorgt ervoor dat de grote van de bandbreedte verdubbeld terwijl de data rate hetzelfde blijft. Echter biedt \gls{ook} geen ondersteuning voor zowel dimming of flicker controle.
	
	\item \textbf{\acrfull{ppm}:} \gls{ppm} is een modulatiemethode waarbij bits gecodeerd worden door een puls op een bepaalde plaats of tijdstip uit te zenden. De symboolduur is onderverdeeld in T verschillende tijdslots en de positie van de puls bepaalt het verzonden symbool. Dit is ook een eenvoudige versie waar veel varianten op bestaan. Zo zijn er \acrfull{vppm}, \acrfull{oppm}, \acrfull{mppm} en nog veel meer. Elk met zijn voordelen en nadelen die afgewogen moeten worden bij het maken van een keuze. 
	
	\item \textbf{\acrfull{ofdm}:} \gls{ofdm} kan gebruikt worden om in het kader van VLC \acrfull{isi} te verminderen en het gebruik van bandbreedte te verbreden. De methode staat toe om data te encoderen op meerdere carrierfrequenties. Deze carriers overlappen elkaar op het frequentiespectrum en voor demodulatie worden er Fast Fourier Transformatie algoritmes gebruikt.
	
	\item \textbf{\acrfull{csk}:} In \gls{csk} varieert de kleuren rood, groen en blauw in een \gls{led} om een signaal te moduleren. Dit maakt het geschikter voor VLC dan gebruikelijke technieken doordat men met een hogere bandbreedte kan werken, en makkelijker dimming en flicker controle kan implementeren.
\end{itemize}

\subsubsection{Type ontvanger}
De ontvanger is een belangrijk onderdeel van het positieneersysteem. Het ontvangen van data zoals ID, positie, signaalsterkte en \gls{toa} zijn afhankelijk van het type ontvanger. Ook de data rate is een belangrijke factor die meespeelt in de keuze. We kunnen de types ontvanger onderverdelen in twee categorie\"en: De photodiode en Image Sensors.
\begin{itemize}
	\item \textbf{\gls{pd}:} De \gls{pd} bestaat uit halfgeleidermateriaal dat invallende fotonen omzet in een elektrische stroom. Deze kunnen in verschillende formaten voorkomen en gevoelig zijn voor bepaalde delen van het optisch spectrum door middel van optische filters. Door de snelle responsie is de \gls{pd} geschikt voor \gls{rss}, \gls{toa} en \gls{tdoa} algoritmes.
	
	\item \textbf{Image Sensors:}  Vergeleken met de \gls{pd} is de Image Sensor een goedkopere optie. De Image Sensor of imager zet lichtgolven om in elektrische stroom.
	\todo{verschil PD en image sensors uitleggen, nog op te zoeken}
\end{itemize}
\subsection{Classificatie}
% classificatie van VLC-based-IPS op 3 keyeigenschappen
\subsubsection{Wiskundige methode}

\begin{itemize}
	\item Proximity
	\item Triangulation
	\begin{itemize}
		\item \acrfull{aoa}
		\item \acrfull{toa}
		\item \acrfull{rss}
	\end{itemize}
	\item Fingerprint
	\begin{itemize}
		\item Map-Based fingerprint
		\item Online stage/runtime stage
	\end{itemize}
\end{itemize}
\subsubsection{Sensor assisted methode}
\subsubsection{Optimization methode}

\newpage
